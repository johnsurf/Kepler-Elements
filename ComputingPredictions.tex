\section{Computation of the Predictions}

Assume the data consists of line-of-sight measurements from an observer whose position as a function of time is $\vec{R}_{obs}(t)$. At each of the measurement times $t_i$, the observer is located at $\vec{R}_i = \vec{R}_{obs}(t_i)$ and makes a measurement of a line-of-sight vector ${\bf l}_i = [x_i, y_i, z_i]$ which we assume has been transformed into the Earth-Centered Intertial (ECI) reference system by a suitable transformation from whatever reference frame the measurement was taken in Earth-Centered Earth Fixed (ECEF) or some other typical observation frame. Assuming we have an initial estimate of the object-of-interest's state vector at a reference time $t_0$ which might be given in position and velocity space or the classical orbital elements or others. We can implement the respective equations of motion to propagate these state vectors as needed to the current time of interest. Obviously whatever state vectors are used as the underlying independent variables, to compute the predicted state which can be compared to the observed data, we must identify the functional form this prediction is to take. As an example, the predicted line-of-sight vector is 
$$\vec{l}_i = \vec{R}(t_i) - \vec{R}_i$$
One can translate this vector into more convenient quantities related to the data format e.g., if the range is not measured, then we can work with $(Az, El)$ or $\theta, \phi$. In terms of spherical polar coordinates this vector's components in ECI are given by 
$$\vec{l} = [l_x, l_y, l_z] = l_{\hbox{mag}} [\sin(\theta)\cos(\phi), \sin(\theta)\sin(\phi), \cos(\theta)],$$where $\l_{\hbox{mag}} = \sqrt{l_x^2 + l_y^2 + l_z^2}$. The unit vector that describes the line-of-sight direction, is given by 
$$\vec{u} = [\sin(\theta)\cos(\phi), \sin(\theta)\sin(\phi), \cos(\theta)]$$and the relationship between these quantities and the components of $\vec{l}$ are given by
\begin{align*}
l_{\hbox{mag}} &= \sqrt{l_x^2 + l_y^2 + l_z^2}\\
\theta               &= \cos^{-1}(l_z/l_{\hbox{mag}})\\
\phi                  &= \hbox{atan2}(l_y,l_x)\\ \numberthis \label{eqnLOS}
\end{align*}Note that atan2$(l_y,l_x)$ takes care of the quadrant determination automatically. If the range rate and rates of change in the angles were available, we could make room for those measurements as well in the above list. Some care may still be needed so that directions near $\phi=0$ and $\phi=2\pi$ are taken to be close to each other and not far apart. Closeness in phi can be determined by using dot products and cross products followed by a further atan2 to pin down the actual relative phi-angle difference in such cases. \\

Ultimately we need to work out the Jacobians that will take us all the way back to the propagated state vectors in such a manner that the LSE method can be iterated and the state vector estimate updated. We shall proceed starting with the items corresponding to the above list and work our way back to the relevant state vector by either application of the chain rule, or by direct computation based on the method of backwards differentiation which determines each Jacobian based directly on the functional dependence according to the respective composition of the functions involved and their direct differentiation.